% This is Main.tex, a sample chapter demonstrating the
% LLNCS macro package for Springer Computer Science proceedings;
% Version 2.20 of 2017/10/04
% All rights to Springer

\documentclass[runningheads]{llncs}

\usepackage{graphicx}
\usepackage{pgfplots}
\usepackage{amsmath}
\usepackage{algorithm}
\usepackage{algpseudocode}
\usepackage{paracol}
\globalcounter{algocf}
\usepackage{wrapfig}
\usepackage{colortbl}

\usepackage{amssymb}
\usepackage{subcaption}

\newcommand{\mymk}[1]{%
  \tikz[baseline=(char.base)]\node[anchor=south west, draw,rectangle, rounded corners, inner sep=2pt, minimum size=4mm,
    text height=2mm](char){\ensuremath{#1}} ;}

\usepackage{xfakebold}

\newcommand{\fbseries}{\unskip\setBold\aftergroup\unsetBold\aftergroup\ignorespaces}
\makeatletter
\newcommand{\setBoldness}[1]{\def\fake@bold{#1}}
\makeatother


\pgfplotsset{width=5cm, compat = 1.17}%
\frenchspacing

\begin{document}

\title{Title} 

\titlerunning{Running Title}


\author{Autthor 1\inst{1}, Author 2\inst{1}, \and Author 3\inst{1,2}}
\institute{Indian Institute of Technology, Madras, Chennai 600036, TN, India\\
                \email{\{roll\_no\_1,roll\_no\_2,roll\_no\_3\}@smail.iitm.ac.in} \and
            University 2
                \email{email}
           }

\authorrunning{Team Number: 123 }
%
\maketitle   % typeset the header of the contribution
%
\begin{abstract}
\input{Sections/Abstract}

\keywords{Keyword 1 \and Keyword 2 \and Keyword 3}
\end{abstract}

\input{Sections/Introduction}

\section{Problem Statement}\label{sec:prob_stat}

\input{Sections/Background}

\input{Sections/Approach}

\section{Experimentation}\label{sec:exp}

\input{Sections/Discussion}

\input{Sections/ConclusionAndFutureWork}

\bibliographystyle{splncs04}
\bibliography{refs}

\end{document}